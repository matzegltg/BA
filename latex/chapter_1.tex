\chapter{Introduction and Problem Definition}
Mainly due to the wide availability of data, high computational power and versatile application possibilities the usage of Artificial Neural Networks (NNs) has increased rapidly in the last years. With more than 1.9 million publications \cite{NumberDNN} including the word \textit{Artificial Neural Network} a hype in using NNs can be observed.\\
Generally NNs are considered as universal function approximators \cite{Cybenko1989Dec}. Based on input data with corresponding targets the NN learns in train passes unknown functional relations and evaluates in test passes input data the model has not seen before. If sufficient data is available, a trained NN is e.g. able to distinguish whether a dog or a cat is to be seen on an image, using the pixels of the image as input \cite{Team2021Aug}.\\
If only few or scarse data is available, the DNN starts to overfit its input and does not learn a generalized model. Physics-informed machine learning \cite{Raissi2017Nov} is a promising way to improve the generalization ability of NNs by providing physical knowledge e.g. in the form of model equations to the NN.\\
In the physics-informed Finite Volume Neural Network (FINN) developed by Karlbauer et al. \cite{karlbauer2021composing}, learning abilities of NNs and knowledge about solving partial differential equations (PDEs) with the Finite Volume (FV) method, can be combined to model e.g. advection-dispersion processes of substances in contaminated soil. Models were trained and tested using both synthetic data, which are spatio-temporal solutions of processes described by PDEs, and experimental data. Experimental information were previously used, to learn constitutive relationships expressed by the retardation factor of a sorption process \cite{Praditia2021Apr}.\\
\\
PFAS (per- and polyfluoroalkyl substances) are a group of approximately 4700 synthetic chemically and thermally stable, water and grease repellent chemicals that are frequently used in industry e.g. in cosmetics, tableware, for surface treatment of metals or in crop protection products. Chemically, PFAS consist of perfluorinated carbon chains, which can accumulate in various organisms due to their longevity. Humans ingest the toxic PFAS primarily from food or drinking water. PFAS can accumulate in the vadose zone, which is the area between the earth's surface and the water table, or enter groundwater. In particular, sorption and reaction processes of PFAS take place in this zone, which are still poorly understood in their entirety \cite{Gellrich2012May, Guo2020Feb}.\\
Prior to this work, experimental data on PFAS transport through soil was generated \cite{Bierbaum2022Mar} including column experiments performed under water-saturated conditions to evaluate the long-term leaching characteristics. Samples of contaminated soil were placed in a column with 2 sand layers (Dorsilit 7) on top and bottom and were flushed with water. Outflowing water was collected in a container in order to determine the concentration of various PFAS after irregular periods of time. In addition, the mass concentration of sorbed PFAS in the soil at the end of the experiment was determined. In particular, the substance perfluorooctanesulfonic acid (PFOS) was investigated in the experiments. The perfluorinated PFAS is, in contrast to polyfluorinated compounds, difficult to degrade \cite{Leung2022Jun}.\\
\\
For modelling non-equilibrium 1D flow and transport processes in soil different models were described previously. There have been introduced different uniform transport \cite{Nkedi-Kizza1984Aug} and non-equilibrium models \cite{VanGenuchten1981, Toride1993Jul}, which are based on the Richards \cite{Richards1931Nov} and Advection-Diffusion (AD) equation. A lot of these models are included in the software package Hydrus 1D of Simunek et al. \cite{Simunek2008May}. Generally the non-equilibrium models distinguish between physical factors and chemical factors. Both models have also been combined to improve the description of solute transport and water flow \cite{Dam1997}.\\
The chemical non-equilibrium Two-Site sorption model (2SS) \cite{SelimH1976, vanGenuchten1989Sep}, assumes two sorption sites: The first site describes sorption processes already in equilibrium and the second a first-order kinetic rate process.\\
\\
Due to the lack of understanding of PFAS processes occurring in vadose zone, but the existence of various model equations to describe the physical and chemical non-equilibrium transport of substances, this topic is ideally suited to predict and learn occuring processes using a physics-informed NN based on scarse experimental data. In this work FINN was extended to learn the transport processes of PFOS through soil including the 2SS model. Using both, information about solving the PDE and experimental data of Bierbaum et al. relevant time dependent sorption processes and spatio-temporal profiles of transport quantities, which cannot be derived directly from experimental data, were learned. The core contributions of this work are:
\begin{itemize}
    \item Implementation and validation of a FD solver for the AD equation including the 2SS model to generate synthetic training data for FINN.
    \item Validated embedding of the AD equation including the 2SS model into the FINN framework.
    \item Application of FINN on synthetic and scarse real-world experimental data in order to understand and generalize transport and sorption processes occuring in contaminated soil.
\end{itemize}